\subsection{Our Dynamic Frontier approach}
\label{sec:frontier}

The \textit{Naive-dynamic} approach processes all vertices in the graph until convergence. However, if a batch update $\Delta^{t-} \cup \Delta^{t+}$ is small compared to the total number of edges $|E|$, then it is expected that the ranks of only a few vertices change. Our proposed \textit{Dynamic Frontier} approach incorporates this aspect, and identifies affected vertices efficiently via an incremental process.

% Incremental Iteration Method for Fast PageRank Computation (2015): In this paper, Kim and Choi \cite{kim2015incremental} propose an asynchronous approach for computing PageRank combined with the standard approach for finding the affected set of vertices (like Desikan et al. \cite{rank-desikan05}).


\subsubsection{Explanation of the approach}
\label{sec:frontier-explanation}

Consider a batch update consisting of edge deletions $(u, v) \in \Delta^{t-}$ and insertions $(u, v) \in \Delta^{t+}$. We first initialize the rank of each vertex to that obtained in the previous snapshot of the graph.

\begin{figure*}[hbtp]
  \centering
  \subfigure[Initial graph]{
    \label{fig:about-df-01}
    \includegraphics[width=0.23\linewidth]{out/about-df-01.pdf}
  }
  \subfigure[Marking affected (initial)]{
    \label{fig:about-df-02}
    \includegraphics[width=0.23\linewidth]{out/about-df-02.pdf}
  }
  \subfigure[After first iteration]{
    \label{fig:about-df-03}
    \includegraphics[width=0.23\linewidth]{out/about-df-03.pdf}
  }
  \subfigure[After second iteration]{
    \label{fig:about-df-04}
    \includegraphics[width=0.23\linewidth]{out/about-df-04.pdf}
  } \\[-2ex]
  % \subfigure[]{
  %   \label{fig:about-df-05}
  %   \includegraphics[width=0.18\linewidth]{out/about-df-05.pdf}
  % }
  \caption{An example explaining the \textit{Dynamic Frontier} approach. Affected vertices are shown with yellow fill, and vertices with a change in rank greater than frontier tolerance $\tau'$ are shown with red border.}
  \label{fig:about-df}
\end{figure*}




% \begin{figure*}[hbtp]
%   \centering
%   \subfigure{
%     \label{fig:about-dynamic-df-detail}
%     \includegraphics[width=0.84\linewidth]{out/about-dynamic-frontier.jpg}
%   } \\[-2ex]
%   \caption{Process of marking affected nodes. Figure (a) on the left is the graph $G^{t-1}$, Figure (b) shows the graph $G^t$ with two new edges, shown in dashed lines, added. Vertices filled in Figure (b) indicate the affected vertices. The numbers on the vertices in Figure (b) indicate the iteration at which the vertex is marked as affected.}
%   \label{fig:about-dynamic-frontier}
% \end{figure*}


\paragraph{Initial marking of affected vertex on edge deletion/insertion:}

For each edge deletion/insertion $(u, v)$, we initially mark the outgoing neighbors of the vertex $u$ in the previous $G^{t-1}$ and current graph snapshot $G^t$ as affected (lines \ref{alg:with-barrier--mark-begin}-\ref{alg:with-barrier--mark-end} in Algorithm \ref{alg:with-barrier}, and lines \ref{alg:barrier-free--mark-begin}-\ref{alg:barrier-free--mark-end} in Algorithm \ref{alg:barrier-free}).

\paragraph{Incremental marking of affected vertices upon change in rank of a given vertex:}

Next, while performing PageRank computation (lines \ref{alg:with-barrier--compute-begin}-\ref{alg:with-barrier--compute-end} in Algorithm \ref{alg:with-barrier}, and lines \ref{alg:barrier-free--compute-begin}-\ref{alg:barrier-free--compute-end} in Algorithm \ref{alg:barrier-free}), if the rank of any affected vertex $v$ changes in an iteration by an amount greater than the \textit{frontier tolerance} $\tau'$, we mark its outgoing neighbors as affected (lines \ref{alg:with-barrier--remark-begin}-\ref{alg:with-barrier--remark-end} in Algorithm \ref{alg:with-barrier}, and lines \ref{alg:barrier-free--remark-begin}-\ref{alg:barrier-free--remark-end} in Algorithm \ref{alg:barrier-free}). This process of marking vertices continues in every iteration.

\begin{algorithm}[!hbt]
\caption{Our parallel Dynamic Frontier PageRank (DF-PageRank).}
\label{alg:prdf}
\begin{algorithmic}[1]
\Require{$\Delta^{t-}, \Delta^{t+}$: Edge deletions and insertions (input)}
\Ensure{$R, R_{new}$: Previous, current rank vector}
\Ensure{$V_A[u]$: Is vertex $u$ affected due to current batch update?}

\Statex

\Function{$\text{DF}_{\text{WBPR}}$}{$G^{t-1}, R^{t-1}, \Delta^{t-}, \Delta^{t+}$}
  \State $V_A \gets \{0\}$ \textbf{;} $R_{new} \gets R \gets R^{t-1}$
  \State $\rhd$ Mark initial affected
  \ForAll{$(u, v) \in \Delta^{t-} \cup \Delta^{t+} \textbf{in parallel}$} \label{alg:with-barrier--mark-begin}
    \ForAll{$v' \in (G^{t-1} \cup G^t).out(u)$}
    \State $V_A[v'] \gets 1$
    \EndFor
  \EndFor \label{alg:with-barrier--mark-end}
  \State \textbf{wait for all threads} (implicit barrier)
  \ForAll{$i \in [0 .. MAX\_ITERATIONS)$} \label{alg:with-barrier--compute-begin}
    \ForAll{$v \in V^t$ \textbf{in parallel}}
      \State $\rhd$ Is vertex not affected?
      \If{$V_A[v] = 0$} \textbf{continue}
      \EndIf
      \State $r \gets (1 - \alpha)/|V^t|$
      \ForAll{$u \in G^t.in(v)$}
        \State $r \gets r + \alpha * R[u] / |G^t.out(u)|$
      \EndFor
      \State $\Delta r \gets |r - R[v]|$ \textbf{;} $R_{new}[v] \gets r$
      \State $\rhd$ Is rank change $>$ frontier tolerance?
      \If{$\Delta r > \tau'$} \label{alg:with-barrier--remark-begin}
        \ForAll{$v' \in G^t.out(v)$} $V_A[v'] \gets 1$
        \EndFor
      \EndIf \label{alg:with-barrier--remark-end}
    \EndFor
    \State \textbf{wait for all threads} (implicit barrier)
    \State $\Delta R \gets l_\infty Norm(R, R_{new})$ \textbf{in parallel}
    \State $swap(R_{new}, R)$
    \State $\rhd$ Ranks converged?
    \If{$\Delta R \le \tau$} \textbf{break}
    \EndIf
  \EndFor \label{alg:with-barrier--compute-end}
  \State \ReturnInline{$R$}
\EndFunction
\end{algorithmic}
\end{algorithm}




%% Requires (parameters):
% G(V, E): a directed unweighted graph
% R: initial ranks (1/N for static)

%% Parameter values:
% MAX\_ITERATIONS = 500
% DAMPING\_FACTOR = 0.85
% TOLERANCE = 10^-10



\subsubsection{A simple example}

Figure \ref{fig:about-frontier} shows an example of the \textit{Dynamic Frontier} approach. The original graph, shown in Figure \ref{fig:about-frontier-01} consists of $16$ vertices. Subsequently, Figure \ref{fig:about-frontier-02} shows a batch update applied to the original graph involving the deletion of an edge from vertex $2$ to $1$ and the insertion of an edge from vertex $4$ to $12$. Following the batch update, we perform the initial step of the \textit{Dynamic Frontier} approach, marking outgoing neighbors of $2$ and $4$ as affected, i.e., $1$, $3$, $4$, $8$, and $12$ are marked as affected. Note that vertex $2$ is not affected as it is a source of the change while vertex $4$ being a neighbour of $2$ is also affected. Now, we are ready to execute the first iteration of PageRank algorithm.

During the first iteration (see Figure \ref{fig:about-frontier-03}), the ranks of affected vertices are updated. Consider that the change in rank of vertices $1$ and $12$ is observed to be greater than frontier tolerance $\tau'$, shown with red border in Figure \ref{fig:about-frontier-03}. In response to this, the \textit{Dynamic Frontier} approach incrementally marks the outgoing neighbors of $1$ and $12$ as affected, specifically vertices $5$, $11$, and $14$.

During the second iteration (see Figure \ref{fig:about-frontier-04}), the ranks of affected vertices are again updated. Further, consider that the change in rank of vertices $3$, $5$, $11$, and $14$ is observed to be greater than frontier tolerance $\tau'$, shown with red border in Figure \ref{fig:about-frontier-03}. As before, we mark the outgoing neighbors of $3$, $5$, $11$, and $14$ as affected, namely vertices $4$, $6$, and $15$.

In the subsequent iteration, the ranks of affected vertices are again updated. If the change in rank of every vertex is within (iteration) tolerance $\tau$, the ranks of vertices have converged, and the algorithm terminates.

% Illustration of the \textit{Dynamic Frontier} approach through a specific example. The initial graph consists of $16$ vertices and $25$ edges. The graph is then updated with an edge insertion $(4, 12)$, and an edge deletion $(2, 1)$. Accordingly, the outgoing neighbors of vertices $4$ ($3$ and $12$) and $2$ ($1$, $4$, and $8$) are marked as affected (shown with yellow fill). When the ranks of these affected vertices are computed in the first iteration, it is found that change in rank of vertices $1$ and $12$ exceeds the frontier tolerance $\tau_f$ (shown with red border). Thus, outgoing neighbors of vertices $1$ ($3$ and $5$) and $12$ ($11$ and $14$) are also marked as affected. In the second iteration, the change in rank of vertices $3$, $5$, $11$, and $14$ is greater than $\tau_f$ --- thus their outgoing vertices are marked as affected. In the subsequent iteration, the ranks of affected vertices are again updated. If the change in rank of every vertex is within (iteration) tolerance $\tau$, the ranks of vertices have converged, and the algorithm terminates.


\subsection{Dynamic Frontier With-barrier PageRank (\FroWbar{})}
\label{sec:frontier-withbarrier}

As mentioned in Section \ref{sec:withbarrier}, we perform a synchronous rank computation with \FroWbar{} using two rank vectors $R$, and $R_{new}$ (see Algorithm \ref{alg:with-barrier}).

\paragraph{Tracking affected vertices:}

To track vertices that are marked as affected due to the current batch update $\Delta^{t-} \cup \Delta^{t+}$, we use a flag vector $V_A$ (an 8-bit integer vector).

\paragraph{Detecting convergence:}

Note that $\Delta r$ represents the change in rank of single vertex, and $\Delta R$ represents the $L_\infty$-norm between the previous $R$ and the current rank vector $R_{new}$. When $\Delta R$ lies within (iteration) tolerance $\tau$, ranks are considered to have converged.



\subsection{Synchronous vs Asynchronous implementation}

In a synchronous implementation, separate input and output rank vectors are used, ensuring deterministic results for parallel algorithms through vector swapping at the end of each iteration. In contrast, an asynchronous implementation utilizes a single rank vector, potentially achieving faster convergence and eliminating memory copies for unaffected vertices in dynamic approaches\ignore{, but introduces non-deterministic results in parallel algorithms}.

To check whether a synchronous or an asynchronous implementation is suitable for \textit{Dynamic Frontier} PageRank, we attempt both the implementations for \textit{Static}, \textit{Naive-dynamic}, \textit{Dynamic Traversal}, and \textit{Dynamic Frontier} PageRank on batch updates (consisting purely of edge insertions) of size $10^{-7}|E|$ to $0.1|E|$. Figure \ref{fig:measure-affected} shows the average relative runtime with asynchronous implementations of each approach compared to their respective synchronous implementations. Results indicate that asynchronous implementations, especially for smaller batch sizes.\ignore{This is due to a somewhat faster convergence and the absence of copy overhead (for \textit{Dynamic Traversal} and \textit{Dynamic Frontier} approaches).} Accordingly, we use the asynchronous implementations of \textit{Naive-dynamic}, \textit{Dynamic Traversal}, and \textit{Dynamic Frontier} PageRank.




\subsection{Determination of Frontier tolerance ($\tau_f$)}

We now measure a suitable value for frontier tolerance $\tau_f$ that allows us to minimize the number of vertices we process (after marking them as affected), while ensuring that we obtain ranks with the desired tolerance, i.e. we obtain ranks with no higher error than \textit{Static} PageRank for the same tolerance setting. For this, we adjust frontier tolerance $\tau_f$ from $\tau$ to $\tau / 10^5$ and obtain ranks of vertices with the \textit{Dynamic Frontier} approach on batch updates (consisting purely of edge insertions) of size $10^{-7}|E|$ to $0.1|E|$.

Figure \ref{fig:adjust-frontier} illustrates the average relative runtime and rank error (in comparison to ranks obtained with reference Static PageRank) using the \textit{Dynamic Frontier} approach. The figure suggests that as $\tau_f$ increases, runtime decreases, but it is accompanied by an increase in error. A frontier tolerance $\tau_f$ set at $\tau/10^4$ or $\tau/10^5$ yields ranks with lower error than \textit{Static} PageRank, making them acceptable for uniformly random batch updates. To err on the side of caution, we opt for a frontier tolerance of $\tau_f = \tau/10^5$.

\begin{figure*}[!hbt]
  \centering
  \subfigure{
    \label{fig:approach-async--mean}
    \includegraphics[width=0.48\linewidth]{out/approach-async-mean.pdf}
  }
  \subfigure{
    \label{fig:approach-async--batch}
    \includegraphics[width=0.48\linewidth]{out/approach-async-batch.pdf}
  } \\[-2ex]
  \caption{Average Relative runtime with asynchronous implementations of \textit{Static}, \textit{Naive-dynamic}, \textit{Dynamic Traversal}, and \textit{Dynamic Frontier} approach compared to their respective synchronous implementations, on batch updates of size $10^{-7}|E|$ to $0.1|E|$ (right), and overall (left). The results indicate that asynchronous implementations are faster than synchronous ones, especially for smaller batch sizes. This is due to a somewhat faster convergence and the absence of copy overhead (for \textit{Dynamic Traversal} and \textit{Dynamic Frontier} approaches).}
  \label{fig:approach-async}
\end{figure*}

\begin{figure}[!hbt]
  \centering
  \subfigure{
    \label{fig:adjust-frontier--runtime}
    \includegraphics[width=0.98\linewidth]{out/adjust-frontier-runtime.pdf}
  } \\[-1ex]
  \subfigure{
    \label{fig:adjust-frontier--error}
    \includegraphics[width=0.98\linewidth]{out/adjust-frontier-error.pdf}
  } \\[-2ex]
\caption{Adjust Frontier. The error is averaged over the graphs in Table \ref{tab:dataset}.}
  \label{fig:adjust-frontier}
\end{figure}





% Dynamic Frontier (DF) approach
% Adjusting tolerance, Frontier tolerance, Mark DelRank / DelContrib
% Dynamic Frontier optimizations
% Edge-balanced approach (Chunk size)
