\subsection{Our Dynamic Frontier approach}
\label{sec:frontier}

The \textit{Naive-dynamic} approach processes all vertices in the graph until convergence. However, if a batch update $\Delta^{t-} \cup \Delta^{t+}$ is small compared to the total number of edges $|E|$, then it is expected that the ranks of only a few vertices change. Our proposed \textit{Dynamic Frontier} approach incorporates this aspect, and identifies affected vertices efficiently via an incremental process. We apply this approach to \textit{With-barrier} and \textit{Barrier-free} PageRank in Algorithms \ref{alg:with-barrier} and \ref{alg:barrier-free}, which we also denote as \FroWbar{} and \FroBarf{} respectively.

% Incremental Iteration Method for Fast PageRank Computation (2015): In this paper, Kim and Choi \cite{kim2015incremental} propose an asynchronous approach for computing PageRank combined with the standard approach for finding the affected set of vertices (like Desikan et al. \cite{rank-desikan05}).


\subsubsection{Explanation of the approach}
\label{sec:frontier-explanation}

Consider a batch update consisting of edge deletions $(u, v) \in \Delta^{t-}$ and insertions $(u, v) \in \Delta^{t+}$. We first initialize the rank of each vertex to that obtained in the previous snapshot of the graph.

\begin{figure*}[hbtp]
  \centering
  \subfigure[Initial graph]{
    \label{fig:about-df-01}
    \includegraphics[width=0.23\linewidth]{out/about-df-01.pdf}
  }
  \subfigure[Marking affected (initial)]{
    \label{fig:about-df-02}
    \includegraphics[width=0.23\linewidth]{out/about-df-02.pdf}
  }
  \subfigure[After first iteration]{
    \label{fig:about-df-03}
    \includegraphics[width=0.23\linewidth]{out/about-df-03.pdf}
  }
  \subfigure[After second iteration]{
    \label{fig:about-df-04}
    \includegraphics[width=0.23\linewidth]{out/about-df-04.pdf}
  } \\[-2ex]
  % \subfigure[]{
  %   \label{fig:about-df-05}
  %   \includegraphics[width=0.18\linewidth]{out/about-df-05.pdf}
  % }
  \caption{An example explaining the \textit{Dynamic Frontier} approach. Affected vertices are shown with yellow fill, and vertices with a change in rank greater than frontier tolerance $\tau'$ are shown with red border.}
  \label{fig:about-df}
\end{figure*}




% \begin{figure*}[hbtp]
%   \centering
%   \subfigure{
%     \label{fig:about-dynamic-df-detail}
%     \includegraphics[width=0.84\linewidth]{out/about-dynamic-frontier.jpg}
%   } \\[-2ex]
%   \caption{Process of marking affected nodes. Figure (a) on the left is the graph $G^{t-1}$, Figure (b) shows the graph $G^t$ with two new edges, shown in dashed lines, added. Vertices filled in Figure (b) indicate the affected vertices. The numbers on the vertices in Figure (b) indicate the iteration at which the vertex is marked as affected.}
%   \label{fig:about-dynamic-frontier}
% \end{figure*}


\paragraph{Initial marking of affected vertex on edge deletion/insertion:}

For each edge deletion/insertion $(u, v)$, we initially mark the outgoing neighbors of the vertex $u$ in the previous $G^{t-1}$ and current graph snapshot $G^t$ as affected (lines \ref{alg:with-barrier--mark-begin}-\ref{alg:with-barrier--mark-end} in Algorithm \ref{alg:with-barrier}, and lines \ref{alg:barrier-free--mark-begin}-\ref{alg:barrier-free--mark-end} in Algorithm \ref{alg:barrier-free}).

\paragraph{Incremental marking of affected vertices upon change in rank of a given vertex:}

Next, while performing PageRank computation (lines \ref{alg:with-barrier--compute-begin}-\ref{alg:with-barrier--compute-end} in Algorithm \ref{alg:with-barrier}, and lines \ref{alg:barrier-free--compute-begin}-\ref{alg:barrier-free--compute-end} in Algorithm \ref{alg:barrier-free}), if the rank of any affected vertex $v$ changes in an iteration by an amount greater than the \textit{frontier tolerance} $\tau'$, we mark its outgoing neighbors as affected (lines \ref{alg:with-barrier--remark-begin}-\ref{alg:with-barrier--remark-end} in Algorithm \ref{alg:with-barrier}, and lines \ref{alg:barrier-free--remark-begin}-\ref{alg:barrier-free--remark-end} in Algorithm \ref{alg:barrier-free}). This process of marking vertices continues in every iteration.

% \input{src/alg-with-barrier}
% \input{src/alg-barrier-free}


\subsubsection{A simple example}

Figure \ref{fig:about-frontier} shows an example of the \textit{Dynamic Frontier} approach. The original graph, shown in Figure \ref{fig:about-frontier-01} consists of $16$ vertices. Subsequently, Figure \ref{fig:about-frontier-02} shows a batch update applied to the original graph involving the deletion of an edge from vertex $2$ to $1$ and the insertion of an edge from vertex $4$ to $12$. Following the batch update, we perform the initial step of the \textit{Dynamic Frontier} approach, marking outgoing neighbors of $2$ and $4$ as affected, i.e., $1$, $3$, $4$, $8$, and $12$ are marked as affected. Note that vertex $2$ is not affected as it is a source of the change while vertex $4$ being a neighbour of $2$ is also affected. Now, we are ready to execute the first iteration of PageRank algorithm.

During the first iteration (see Figure \ref{fig:about-frontier-03}), the ranks of affected vertices are updated. Consider that the change in rank of vertices $1$ and $12$ is observed to be greater than frontier tolerance $\tau'$, shown with red border in Figure \ref{fig:about-frontier-03}. In response to this, the \textit{Dynamic Frontier} approach incrementally marks the outgoing neighbors of $1$ and $12$ as affected, specifically vertices $5$, $11$, and $14$.

During the second iteration (see Figure \ref{fig:about-frontier-04}), the ranks of affected vertices are again updated. Further, consider that the change in rank of vertices $3$, $5$, $11$, and $14$ is observed to be greater than frontier tolerance $\tau'$, shown with red border in Figure \ref{fig:about-frontier-03}. As before, we mark the outgoing neighbors of $3$, $5$, $11$, and $14$ as affected, namely vertices $4$, $6$, and $15$.

In the subsequent iteration, the ranks of affected vertices are again updated. If the change in rank of every vertex is within (iteration) tolerance $\tau$, the ranks of vertices have converged, and the algorithm terminates.




\subsection{Dynamic Frontier With-barrier PageRank (\FroWbar{})}
\label{sec:frontier-withbarrier}

As mentioned in Section \ref{sec:withbarrier}, we perform a synchronous rank computation with \FroWbar{} using two rank vectors $R$, and $R_{new}$ (see Algorithm \ref{alg:with-barrier}).

\paragraph{Tracking affected vertices:}

To track vertices that are marked as affected due to the current batch update $\Delta^{t-} \cup \Delta^{t+}$, we use a flag vector $V_A$ (an 8-bit integer vector).

\paragraph{Detecting convergence:}

Note that $\Delta r$ represents the change in rank of single vertex, and $\Delta R$ represents the $L_\infty$-norm between the previous $R$ and the current rank vector $R_{new}$. When $\Delta R$ lies within (iteration) tolerance $\tau$, ranks are considered to have converged.




\subsection{Determination of Frontier tolerance ($\tau'$)}

Our experiments show that a frontier tolerance of $\tau' = \tau/1000$, where $\tau$ is the (iteration) tolerance, provides a good speedup with a maximum error of $10^{-9}$ at a batch size of $10^{-4} |E|$ compared to a mean error of $5 \times 10^{-10}$ for \NaiWbar{} and \NaiBarf{} with respect to ranks obtained from reference PageRank (see Section \ref{sec:measurement}). Thus, if an edge in the batch update affects the rank of vertex $u$ (by a small amount), directly or indirectly, all its outgoing neighbors $v' \in G^t.out(u)$ will also be marked as affected, as they are likely to have a change in rank as well.

\begin{figure*}[!hbt]
  \centering
  \subfigure{
    \label{fig:approach-async--mean}
    \includegraphics[width=0.48\linewidth]{out/approach-async-mean.pdf}
  }
  \subfigure{
    \label{fig:approach-async--batch}
    \includegraphics[width=0.48\linewidth]{out/approach-async-batch.pdf}
  } \\[-2ex]
  \caption{Average Relative runtime with asynchronous implementations of \textit{Static}, \textit{Naive-dynamic}, \textit{Dynamic Traversal}, and \textit{Dynamic Frontier} approach compared to their respective synchronous implementations, on batch updates of size $10^{-7}|E|$ to $0.1|E|$ (right), and overall (left). The results indicate that asynchronous implementations are faster than synchronous ones, especially for smaller batch sizes. This is due to a somewhat faster convergence and the absence of copy overhead (for \textit{Dynamic Traversal} and \textit{Dynamic Frontier} approaches).}
  \label{fig:approach-async}
\end{figure*}

\begin{figure}[!hbt]
  \centering
  \subfigure{
    \label{fig:adjust-frontier--runtime}
    \includegraphics[width=0.98\linewidth]{out/adjust-frontier-runtime.pdf}
  } \\[-1ex]
  \subfigure{
    \label{fig:adjust-frontier--error}
    \includegraphics[width=0.98\linewidth]{out/adjust-frontier-error.pdf}
  } \\[-2ex]
\caption{Adjust Frontier. The error is averaged over the graphs in Table \ref{tab:dataset}.}
  \label{fig:adjust-frontier}
\end{figure}



% However, our experiments show that the \textit{Dynamic Traversal} approach does not perform better than the \textit{Naive-dynamic} approach for any batch size. The overhead of this approach, due to several traversals required to identify the affected vertices, limits the performance of this approach.




% Dynamic Frontier (DF) approach
% Adjusting tolerance, Frontier tolerance, Mark DelRank / DelContrib
% Dynamic Frontier optimizations
% Edge-balanced approach (Chunk size)
