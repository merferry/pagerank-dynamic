Parallel algorithms for problems on dynamic graphs have been attracting a lot of research attention in recent years. Examples include the work of Regunta et al. \cite{cent-regunta21}, Banerjee et al. \cite{rank-sahu22}, Haryan et al. \cite{cc-haryan22}, and Khanda et al. \cite{hipc22}. Some of the early work on dynamic graph algorithms in the sequential setting include the seminal sparsification method of Eppstein et al. \cite{graph-eppstein97} and the bounded incremental computation idea of Ramalingam \cite{incr-ramalingam96}. The latter advocates measuring the work done as part of the update in proportion to the effect the update has on the computation.

PageRank is a fundamental algorithm used to measure relationships among vertices and subsets of nodes in various applications. It has been implemented on multicore CPUs \cite{rank-garg16}, GPUs \cite{rapids}, FPGAs \cite{rank-guoqiang20}, SpMV ASICs \cite{rank-sadi18}, CPU-GPU hybrids \cite{rank-giri20}, CPU-FPGA hybrids \cite{rank-li21}, and distributed systems \cite{rank-sarma13}. Dynamic PageRank algorithms aim to handle changes in the input graph efficiently. Zhang \cite{rank-zhang17} presents a simple incremental Pagerank computation system for dynamic graphs on hybrid CPU and GPU platforms that incorporates the Update-Gather-Apply-Scatter (UGAS) computation model. Another approach involves identifying the affected region through BFS or DFS traversal and computing PageRanks only for that region \cite{rank-desikan05, rank-giri20}. Fast Pessimistic PageRank (FPPR) is capable of regenerating scores on node and link addition/deletion, but is based on randomized Monte Carlo (MC) algorithm that provides an estimation of PageRank \cite{rank-pashikanti22}.

Further, Bahmani et al. \cite{rank-bahmani12} propose an algorithm to selectively crawl a small portion of the web to provide an estimate of true PageRank of the graph at that moment, while Berberich et al. \cite{rank-berberich07} present a method to compute normalized PageRank scores that are robust to non-local changes in the graph. Their approaches are orthogonal to our \textit{Dynamic Frontier} approach which focuses on the computation of the PageRank vector itself, not on the process of crawling the web or maintaining normalized scores.

Banerjee et al. \cite{rank-sahu22} show how to obtain the PageRank values of nodes in a dynamic graph under batch updates. The algorithm from \cite{rank-sahu22} uses a strongly-connected component based decomposition of the graph. It limits the computation to strongly connected components that are reachable from any endpoint of the edges in a batch of updates. As the algorithm from \cite{rank-sahu22} uses a strict ordering in the computation involved, the algorithm is not tolerant of thread failures and delays. %They improve on the results of Giri et al. \cite{rank-giri20}.
%% he: need to update

Herlihy and Shavit \cite{Herlihy+:nature:opodis:2011} emphasize the need for lock-free data-structures algorithms that are resilient to arbitrary thread slowdown and failures in parallel computing environments. Several lock-free data-structures have been developed in the literature from stacks, queues to concurrent graphs. Some of these have been explained Herlihy et al. in their book \cite{Herlihy+:Art:Book:2020}. Examples of such algorithms include Compare-and-Swap (CAS) and Read-Copy-Update (RCU), which enable atomic operations on shared variables and efficient read-side access to shared data structures, respectively. These examples demonstrate the feasibility of lock-free parallel algorithms that can withstand thread failures, emphasizing the importance of this approach for building reliable and efficient parallel systems.

Graph analytics systems currently use bulk synchronous parallel (BSP), asynchronous parallel (AP), and matrix-based barrierless asynchronous parallel (MBAP) models \cite{graph-luo20}. Checkpointing is a valuable strategy for handling node failures in iterative graph processing algorithms, but it has some disadvantages, such as high overhead costs \cite{graph-xu16, graph-xu18, graph-yan16}. Optimized checkpointing strategies have been proposed, such as selective checkpointing \cite{graph-wang17}, lightweight checkpoint \cite{graph-yan16}, non-blocking checkpoint \cite{graph-xu16}, dynamically adjusted checkpoint intervals \cite{graph-wang17}, and injected checkpoints \cite{graph-xu16}. In contrast, checkpoint-free strategies exploit algorithmic properties to achieve failure recovery \cite{graph-xu18}.

Some of the existing papers \cite{yang2020graphabcd, sha2017technical, liu2019simd, yu2022toward} address challenges like low convergence, improper load balancing in barrier-based computations. Cong and Bader \cite{lock-cong05} propose parallel lock-free Shiloach-Vishkin algorithm for finding connected components, and parallel block-free (spinlock based) Boruvka algorithm for finding minimum spanning tree. They show that lock-free implementations can handle large graphs and have superior performance.






%% We use an asynchronous approach:
% Real-Time PageRank on Dynamic Graphs (2023): In this paper, Sallinen et al. \cite{sallinen2023real} compute PageRank asynchronously for real-time, on demand PageRank computation with arbitrary granularity. They model PageRank as a flow problem, where mass is absorbed by the page, and the rest is distributed to neighbors. This is done by sending delta values of probability mass depending on edge deletion or insertions by adjustment upon earlier values. Sink/dangling vertices (dead ends) are handled as usual (teleport).

%% Interesting approach:
% PageRank Algorithm Based on Dynamic Damping Factor (2023): Existing methods often set the damping factor empirically, overlooking the relevance of web visitors’ topics. HaoLin et al. \cite{haolin2023pagerank} propose an adaptive dynamic damping factor based on the web browsing context, and demonstrate that it effectively mitigates the impact of noisy web pages on query results and improves the convergence speed.

%% Sliding window approach.
Time-Aware Ranking in Dynamic Citation Networks (2011): In this paper, Ghosh et al. \cite{ghosh2011time} consider the temporal order of links and chains of links connecting to a node with some temporal decay factors, and show that it is more appropriate for predicting future citations and PageRank scores with regard to new citations.

%% Similar to STIC-D:
% Divide and conquer approach for efficient pagerank computation (2006): In this paper, Desikan et al. \cite{desikan2006divide} propose a graph-partitioning technique for PageRank, on which computation can be performed independently.

%% Similar to STIC-D:
% A componentwise PageRank algorithm (2015): In this paper, Engstrom et al. \cite{engstrom2015componentwise} propose two PageRank algorithms, one similar to the Lumping algorithm proposed by Qing et al. which handles certain types of vertices faster, and last, another PageRank algorithm which can handle more types of vertices as well as strongly connected components more effectively. This is similar to the work of Garg et al.

%% Applications of PageRank:
% PageRank Tracker: From Ranking to Tracking (2013): PageRank has been used by Gong et al. \cite{gong2013pagerank} in video object tracking to improve its robustness, i.e., to address difficulties with adaptation to environmental or target change. Determining the target is equivalent to finding the unlabeled sample that is the most associated with the existing labeled set.

Local methods for estimating pagerank values (2004): In this paper, Chen et al. \cite{chen2004local} address online analysis of link evolution by studying several methods for efficiently estimating the PageRank score of a particular web page using only a small subgraph of the entire web.

Abstracting PageRank To Dynamic Asset Valuation (2006): In this paper, Sawilla \cite{sawilla2006abstracting} uses (weighted) PageRank to quickly and dynamically calculate a relative value for all assets in an organization in any context in which dependencies may be specified. Their scheme works in general and will provide asset valuation in any context, be it confidentiality, integrity, availability, or even political capital.

Adaptive Implementation to Update Page Rank on Dynamic Networks (2021): In this oral presentation, Srinivasan \cite{srinivasan2021adaptive} talk about the fact that There are a lot of attempts made to parallelize the page rank algorithm for static networks, however, there are only very few attempts made to compute page rank on dynamic networks. As the networks change with time, computing page rank or updating is an expensive operation, the previous attempts have only approximated the metric to avoid recomputation. In this paper, we introduce a framework where we try to update the page rank of the vertices which embraces change as the network changes. The proposed framework is implemented on a shared memory system and experiments on real-world and synthetic networks show good scalability. The framework proposed gets an input set of networks, initial page rank values for all the vertices, and a set of batches that has the changeset. As the batches are processed in parallel, affected vertices are identified and marked for an update, once the batch is processed the vertices affected or identified their page rank values are computed. The main contribution of this paper is the proposed framework avoids recomputation of all vertices, and only recomputes few vertices, and avoids approximation to provide accurate values.

% Estimating PageRank on graph streams (2011): In this paper, Sarma et al. \cite{rank-sarma11} study the streaming model for PageRank, which uses a small amount of memory (preferably sub-linear in the number of nodes n). They compute approximate PageRank values in Õ(nM−1/4) space and Õ(M3/4) passes. They also give another approach to approximate the PageRank values in just Õ(1) passes although this requires Õ(nM) space.

Efficient PageRank Tracking in Evolving Networks (2015): In this paper, Ohsaka et al. \cite{ohsaka2015efficient} propose a method for locally updating personalized PageRank using the Gauss-Southwell method, where the vertex with the greatest residual is updated first. This is similar to our algorithm.

PageRank on an evolving graph (2012): In this paper, Bahmani et al. \cite{bahmani2012pagerank} propose an algorithm that, at any moment in the time and by crawling a small portion of the graph, provides an estimate of the PageRank that is close to the true PageRank of the graph at that moment.

Fast Incremental PageRank on Dynamic Networks (2019): In this paper, Zhan et al. \cite{zhan2019fast} propose a Monte Carlo based algorithm for PageRank tracking on dynamic networks. Their algorithm maintains nR random walk segments (R random walks starting from each node) in memory.

A Dynamical System for PageRank with Time-Dependent Teleportation (2014): In this paper, Gleich and Rossi \cite{gleich2014dynamical} propose a time-dependent teleportation to the PageRank score. The magnitude of the deviation from a static PageRank vector is given by a PageRank problem with complex-valued teleportation parameters. They demonstrate the utility of dynamic teleportation on both the article graph of Wikipedia, where the external interest information is given by the number of hourly visitors to each page, and the Twitter social network, where external interest is the number of tweets per month. They show that using information from the dynamical system helps improve a prediction task and identify trends in the data.

Temporal PageRank (2016): In this paper, Rozenshtein and Gionis \cite{rozenshtein2016temporal} propose an extension of static PageRank to temporal PageRank so that only temporal walks are considered instead of all possible walks. In order to compute temporal PageRank we need to process the sequence of interactions and calculate the weighted number of temporal walks. Their algorithm counts explicitly the weighted number of temporal walks.

Incremental Iteration Method for Fast PageRank Computation (2015): In this paper, Kim and Choi \cite{kim2015incremental} propose an asynchronous approach for computing PageRank combined with the standard approach for finding the affected set of vertices (like Desikan et al. \cite{rank-desikan05}).

Fast Incremental and Personalized PageRank (2010): In this paper, Bahmani et al. \cite{bahmani2010fast} analyze the efficiency of Monte Carlo methods for incremental computation of PageRank, and personalized PageRank.
